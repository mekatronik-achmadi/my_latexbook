\documentclass[12pt]{article}
\usepackage[utf8]{inputenc}
\usepackage[T1]{fontenc}
\usepackage{mathptmx}
\usepackage{geometry}
\usepackage{mathtools}
\usepackage[english]{babel}
\usepackage{graphicx}
\usepackage[figurename=Gambar]{caption}
\usepackage{hyperref}
\usepackage{minted}
\usepackage{setspace}
\usepackage{color}
\usepackage[explicit]{titlesec}
\usepackage{tocloft}
\usepackage{titletoc}
\usepackage{indentfirst}
\usepackage{caption}
\usepackage{amsmath} 

%\color{white}
%\pagecolor{black}

\hypersetup{colorlinks,
	citecolor=black,
	filecolor=black,
	linkcolor=black,
	urlcolor=black
}

\geometry{
	a4paper,
	left=40mm,
	right=30mm,
	top=20mm,
	bottom=20mm,
}

\date{}

%=============================================================================
% Modified Items

\providecommand{\keywordid}[1]{\textit{Kata Kunci: } #1}
\providecommand{\keyworden}[1]{\textit{Keywords: } #1}

\titleformat{\section}{}{}{0pt}{}
\titleformat{\subsubsection}{\small \bfseries}{\thesubsection}{0pt}{}

\renewcommand{\cftsecleader}{\cftdotfill{\cftdotsep}}
\renewcommand{\cftsubsecleader}{\cftdotfill{\cftdotsep}}

\makeatletter
\let\latexl@section\l@section
\def\l@section#1#2{\begingroup\let\numberline\@gobble\latexl@section{#1}{#2}\endgroup}
\makeatother

\addto\captionsenglish{\renewcommand{\contentsname}{}}
\addto\captionsenglish{\renewcommand{\listfigurename}{}}

\renewcommand{\cftsecfont}{\normalfont}
\renewcommand{\cftsecpagefont}{\normalfont}

\renewcommand{\thefigure}{\arabic{section}.\arabic{figure}}

%=============================================================================
% Document Part

\begin{document}
	
\onehalfspacing
	
%=============================================================================
% Abstrak Indonesia
	
	\pagenumbering{roman}
	\setcounter{page}{2}
	
	\section{Abstrak}
	
	\begin{center}
		\large{\textbf{STUDI EKSPERIMENTAL \textit{MULTI-POINT INTRUSION DETECTOR} BERBASIS OPTICAL TIME DOMAIN REFLECTOMETRY}}
	\end{center}

	\vspace{10pt}
	
	\begin{flushleft}
		Nama Mahasiswa : Achmadi \\
		NRP \hspace{62pt}: 24 16 201 010 \\
		Pembimbing \hspace{24pt}: Agus Muhamad Hatta, S.T, M.Si, Ph.D \\
	\end{flushleft}
	
	\vspace{10pt}

	\renewcommand{\abstractname}{ABSTRAK}
	\begin{abstract}
		Penyusup \textit{(intruder/attacker)} merupakan sebuah aktifitas yang tidak bisa dianggap untuk tidak mungkin terjadi apabila berkaitan dengan bangunan, area (perimeter), maupun objek lain dimana memiliki suatu nilai baik militer, ekonomi, maupun industri. 
		Tindak penyusupan memiliki similaritas dengan tindak pencurian dan menjadi awal dari tindak kriminal yang lebih jauh.
		Tindak penyusupan melalui proses melewati sistem pengawasan atau penjagaan.
		Untuk pencegahannya, maka sistem deteksi penyusup telah menjadi bagian penting dalam banyak sistem keamanan.
		Dibutuhkan sistem deteksi yang akurat dan respon yang cepat. Serat optik telah dikenal mampu menjadi transmisi data maupun sebagai sensor.
		Serat optik dapat digunakan sebagai distributed sensor yang mampu menggantikan banyak sensor tipe titik.
		OTDR telah dikenal sebagai salah satu metode karakteriasi serat optik.
		Melalui OTDR akan didapatkan events yang terjadi pada serat optik secara \textit{real-time}.
		Hasil \textit{trace} OTDR dapat digunakan untuk deteksi penyusup.
		Penelitian ini bertujuan untuk merancang sistem deteksi penyusup menggunakan serat optik dan OTDR yang mampu menemukan penyusup dalam multi-point.
		Dalam penelitian ini diusulkan sebuah studi eksperimental untuk mendapatkan rancang bangun sistem sensor terdistribusi berbasis serat optik dan OTDR untuk deteksi penyusup.
		Studi eksperimental disini divariasikan baik konfigurasi serat optik itu sendiri dan juga divariasikan bentuk distribusi sensor.
		Untuk mendapatkan hasil trace maka digunakan modul Mini-OTDR Anritsu MU909015C, sedangkan untuk eksperimen akan dibangun konstruksi pagar yang akan menjadi tempat instalasi sensor.
		Sebagai pengganti tindak intrusi maka diberikan gaya tekan kepada sensor dengan nilai dan jumlah yang telah ditentukan.
		Luaran yang diharapkan dari penelitian ini adalah hasil rancang bangun sensor instrusi terdistribusi dan rancang bangun algoritma untuk mendapatkan posisi multi-point dari tindak intrusi.   
	\end{abstract}

	\keywordid{Serat optik, OTDR, deteksi penyusup, trace, events, sensor terdistribusi}
	
%=============================================================================
\newpage
\thispagestyle{plain}
\mbox{}
	
%=============================================================================
% Abstrak English
	\newpage
	
	\section{Abstract}
	
	\begin{center}
		\large{\textbf{EXPERIMENTAL STUDY OF MULTI-POINT INTRUSION DETECTOR BASED ON OPTICAL TIME DOMAIN REFLECTOMETRY}}
	\end{center}
	
	\vspace{10pt}
	
	\begin{flushleft}
		Nama Mahasiswa : Achmadi \\
		NRP \hspace{62pt}: 24 16 201 010 \\
		Pembimbing \hspace{24pt}: Agus Muhamad Hatta, S.T, M.Si, Ph.D \\
	\end{flushleft}
	
	\vspace{10pt}
	
	\renewcommand{\abstractname}{ABSTRACT}
	\begin{abstract}
		Intrusion by an intruder is an act that cannot be ignored due to a building, an area (perimeter), or any object regarding it’s either milltary, economical, or industrial values.
		Intrusion has high similarity to thieft crime and can be root of more crime act.
		Intrusion is an act that by-passing any survelleince or security system.
		For prevention, a detection system become essential to many security system.
		This detection system has to be highly accurate and fast respons.
		Optical fiber already known for it’s capabilities to both data transmission and as a distributed sensor.
		An distributed sensor mean single optical fiber section can replace many point-type sensors.
		OTDR already known as one of optical fiber characterization.
		Through OTDR, an events that occur on a optical fiber can be aqcuired by real-time.
		An trace result of OTDR can be used to intrusion detector.
		This research purposes is to get designs of an intrusion detector system using optical fiber and OTDR that can detect any multi-point intrusion.
		This research proposed an experimental study to get an intrusion detector using distributed sensor based optical fiber and OTDR.
		This experimental study proposed to test sensor variation in both optical fiber configuration and distributed sensor shape.
		To get trace result, this research use Anritsu MU909015C Mini OTDR module and for experiment, a fence construction is proposed as distributed sensor placement.
		As intrusion act, this research proposed to give a mechanical pressure to sensor with certain amount and magnitude.
		The expected output of this research are design of distributed sensor for intrusion detection and algorithm to get multi-point intrusion positions.
	\end{abstract}
	
	\keyworden{optical fiber, OTDR, instrusion detection, trace, events, distributed sensor}
	
%=============================================================================
\newpage
\thispagestyle{plain}
\mbox{}

%=============================================================================
% Halaman Pengesahan
\newpage

	\section{Lembar Pengesahan}
	
	\begin{center}
		\textbf{LEMBAR PENGESAHAN}
	\end{center}
	
	\begin{center}
		\textbf{DRAFT PROGRES I}
	\end{center}

	\vspace{10pt}

	\begin{flushleft}
		Judul	: \textbf{STUDI EKSPERIMENTAL MULTI-POINT INTRUSION DETECTOR BERBASIS OPTICAL TIME DOMAIN REFLECTOMETRY}
	\end{flushleft}

	\begin{flushleft}
		Oleh : Achmadi
	\end{flushleft}

	\begin{flushleft}
		NRP : 24 16 201 010
	\end{flushleft}

	\vspace{20pt}
	
	\begin{center}
		\textbf{Telah diseminarkan pada :}
	\end{center}

	\vspace{10pt}

	\begin{flushleft}
		Hari \hspace{17pt}: \\
		Tanggal :\\
		Tempat \hspace{3pt}:  \\
	\end{flushleft}

	\vspace{20pt}
	
	\begin{center}
		\textbf{Mengetahui / menyetujui :}
	\end{center}

	
	\begin{center}
		Dosen Penguji \hspace{150pt} Dosen Pembimbing \\
		Prof. Dr. Ir. Sekartedjo, M.Sc \hspace{75pt} Agus M. Hatta, S.T, M.Si, Ph.D \\
		NIP. 19500402 1979 01 1 001 \hspace{85pt} NIP. 19780902 2003121 002 \\
	\end{center}

	\vspace{100pt}	
	
	\begin{flushleft}
		Dr. rer. nat. Ir. Aulia M.T. Nasution., M.Sc \\
		NIP. 19671117 199702 1 001
	\end{flushleft}

%=============================================================================
\newpage
\thispagestyle{plain}
\mbox{}

%=============================================================================
% Daftar Isi
\newpage

	\begin{center}
		\textbf{{\large Daftar Isi}}
	\end{center}
	
	\tableofcontents

%=============================================================================
\newpage
\thispagestyle{plain}
\mbox{}

%=============================================================================
% Daftar Gambar
\newpage

	\begin{center}
		\textbf{{\large Daftar Gambar}}
	\end{center}

	\listoffigures

%=============================================================================
\newpage
\thispagestyle{plain}
\mbox{}

%=============================================================================
% Daftar Tabel
\newpage

	\begin{center}
		\textbf{{\large Daftar Tabel}}
	\end{center}

%=============================================================================
\newpage
\thispagestyle{plain}
\mbox{}

%=============================================================================
% BAB I
\newpage

	\pagenumbering{arabic}
	\setcounter{page}{9}

	\setcounter{section}{0}
	\section{Pendahuluan}
	
	\begin{center}
		{\large \textbf{BAB I}} \\
		{\large \textbf{Pendahuluan}}
	\end{center}
	
	\subsection{Latar Belakang}
	
	Penyusupan (intrusion) merupakan sebuah aktifitas yang tidak bisa dianggap untuk tidak mungkin terjadi apabila berkaitan dengan bangunan, area (perimeter), maupun objek lain dimana memiliki suatu nilai baik militer, ekonomi, maupun industri\cite{Assets}.
	Pembahasan mengenai penyusup ini erat kaitan dalam pembahasan pencurian (theft) dalam bidang kriminologi, sehingga perilaku penyusupan memiliki kemiripan yang tinggi dengan fenomena pencurian \cite{Felson1998}.
	Kemiripan ini menempatkan definisi penyusupan tidak jauh terhadap definisi pencurian. Secara umum, penyusupan maupun pencurian adalah tindak kriminal yang melibatkan proses melewati maupun menerobos suatu sistem penjagaan atau pengawasan \cite{Chapman}.
	Menurut statistik international, tindak kriminal pencurian memang tidaklah setinggi kriminal lain yang berkaitan pembunuhan dan obat-terlarang \cite{Frate2010}.
	Namun demikian tetap dilakukan pencegahan karena penyusupan adalah awal dari beragam tindak kriminal lebih lanjut \cite{Nesbit}.
	
	Sistem pendeteksi penyusup saat ini telah mengalami perkembangan signifikan.
	Metode konvensional seperti patroli rutin/mendadak kini mulai terganti dengan sistem terintegrasi semisal \textit{Motion Detector}, kamera pengintai, atau pagar listrik \cite{AFL2011}.
	Salah satu metode baru adalah dengan menerapkan teknologi radar untuk mendapatkan objek-objek sekitar perimeter termasuk manusia \cite{Cory1998}.
	Metode lain dalam deteksi penyusup adalah menggunakan propagasi gelombang radio (wireless) untuk mendapatkan gangguan (disturbance) yang diakibatkan oleh penyusup \cite{Elmorsy}\cite{Elmorsy2014}.
	
	Sensor serat optik memiliki potensi besar untuk mendeteksi adanya tindak penyusupan dalam suatu area perimeter sebagaimana serat optik sendiri telah digunakan baik untuk bidang komunikasi dan juga sebagai sensor.
	Serat optik dapat menerima informasi baik secara spasial maupun temporal di sepanjang serat optik \cite{Rao2008}.
	Penggunaan serat optik sebagai sensor sangat tepat karena serat optik tahan terhadap gangguan elektromagnet dan dapat bekerja di lingkungan yang berbahaya \cite{Bremer2016}.
	
	Sistem pendeteksi penyusup dengan berbasis serat optik juga telah banyak menarik perhatian untuk dilakukan riset dan pengembangan disebabkan penggunaannya yang versatile, untuk perlindungan pemukiman, perlindungan sistem komunikasi, atau untuk monitoring sistem perpipaan \cite{Lai2017}.
	istem pendeteksi penyusup menjadi sangat dibutuhkan apabila jika dihadapkan pada kebutuhan keamanan pada bangunan-bangunan krusial \cite{Quwaider2017}.
	Pentingnya keberadaan sistem keamaan yang baik, sehingga diperlukan sistem untuk mendeteksi adanya tindak penyusupan dalam satu perimeter \cite{Huang2017}.
	
	Penggunaan Optical Time Domain Reflectometry (OTDR) telah banyak digunakan untuk karakterisasi suatu bagian tertentu dari serat optik \cite{Dong2015}.
	Karakterisasi menggunakan OTDR memberikan hasil yang dengan tingkat keakuratan dan tingkat kepresisian yang tinggi \cite{He2016}.
	Dengan teknologi OTDR, dapat dilakukan pengawasan secara real-time terhadap semua event yang dikenakan kepada suatu bagian tertentu dari serat optik dengan jangkauan panjang \cite{Optical2007}.
	Saat ini OTDR telah menjadi bagian penting dari sistem komunikasi serat optik yang memiliki peran penting dari segi perawatan (maintenance) maupun pengecekan instalasi jaringan komunikasi serat optik \cite{Nettest2000}.
	
	Penggunaan teknik OTDR secara konvensial saat ini menyediakan karakterisasi serat optik berdasarkan hasil analisa terhadap nilai daya hamburan balik (backscattering) dimana didapatkan titik anomali dalam proses trace \cite{Dong2015}.
	Dalam proses trace, apabila terdapat 2 atau lebih gangguan yang terjadi secara bersamaan, maka hanya gangguan yang paling dekat dengan near-end  yang akan terlihat adanya anomali, sedangkan yang lebih jauh tidak tampak \cite{Bao2012}.
	Hal ini menyebabkan deteksi penyusup yang ada saat ini lebih bersifat single-point dalam satu waktu. 
	
	Dalam penelitian ini akan dilakukan suatu studi eksperimental untuk mendapatkkan rancang bangun serat sesnsor optik untuk mampu mendeteksi adanya penyusupan.
	Selain itu diusulkan pula rancang bangun algoritma untuk mendapatkan metode baru yang dapat diimplementasikan sehingga bisa dilakukan deteksi penyusup secara multi-point.
	
	
	
	\subsection{Rumusan Masalah}
	
	Berdasarkan latar belakang tersebut, maka permasalahan yang akan dikaji dalam penelitian ini adalah sebagai berikut :
	
	\begin{enumerate}
		\item Bagaimana konfigurasi sensor serat optik berbasis singlemode dan multimode untuk mendeteksi adanya intrusi?
		\item Bagaimana pengaturan OTDR yang efektif untuk mendeteksi adanya intrusi?
	\end{enumerate}



	\subsection{Tujuan Penelitian}

	Berdasarkan latar belakang tersebut, maka permasalahan yang akan dikaji dalam penelitian ini adalah sebagai berikut :
	
	\begin{enumerate}
		\item Mendapatkan konfigurasi sensor serat optik berbasis  singlemode dan multimode untuk mendeteksi adanya intrusi
		\item Mendapatkan pengaturan OTDR yang efektif untuk mendeteksi adanya intrusi. 
	\end{enumerate}


	\subsection{Manfaat Penelitian}
	
	Studi eksperimental ini diharapkan dapat meningkatkan tingkat akurasi sistem deteksi penyusup berbasis serat optik terhadap gangguan multi-point melalui produk baik software antar muka (interface), pustaka (libraries) maupun hardware sebagai tambahan (add-on) yang dapat diimplementasikan kepada sistem OTDR yang telah tersedia di pasaran.
	
%=============================================================================
\newpage
\thispagestyle{plain}
\mbox{}

%=============================================================================
% Daftar Pustaka
\newpage

	\section{Kajian Pustaka}
	
	\begin{center}
		{\large \textbf{BAB II}} \\
		{\large \textbf{Kajian Pustaka}}
	\end{center}

	\subsection{Instrusi dan Keamanan}
	
	Intrusi (intrusion) adalah sebuah fenomena dimana sebuah objek melintasi suatu area yang secara hukum terlarang untuk dilintasi.
	Intrusi ini sering terjadi pada bangunan-bangunan yang krusial dan kritikal \cite{Quwaider2017}.
	Pengertian intrusi disini juga diartikan sebagai gangguan terhadap suatu area yang seharusnya tidak ada gangguan, dimana tujuan utama intrusi adalah melewati sistem penjagaan atau keamanan \cite{Chapman}.
	Intrusi yang dimaksud disini bukanlah intrusi dalam artian dalam bidang tektonik maupun bidang keamanan jaringan sistem informasi.
	
	Terdapat ragam jenis intrusi, namun dalam penelitian ini diambil intrusi yang dilakukan dengan menembus batas perimeter berupa pagar.
	Tindak intrusi disini terbagi menjadi lompatan (jump), pendakian (climbing), pemotongan (cutting), menggoyang (waggling), maupun pemukulan (knocking) yang ditunjukkan pada gambar 2.1 \cite{Huang2017}.

	\begin{figure}[h!]
		\centering
		\captionsetup{justification=centering}
   		\includegraphics[width=0.7\linewidth]{images/Bab_2/bab_2_1}
		\caption[Ragam Intrusi]{\small{Sebagian ragam bentuk tindak intrusi}}
	\end{figure}

	\subsection{Serat Optik}
	
	Serat optik merupakan pemandu gelombang silindris dielektrik yang terbuat dari material low-loss seperti plastik maupun gelas silika.
	Serat optik terdiri dari core dimana cahaya dipandu, dan cladding sebagai sebagai selubung core.
	Core memiliki indeks bias lebih tinggi daripada cladding.
	
	\begin{figure}[h!]
		\centering
		\captionsetup{justification=centering}
		\includegraphics[width=0.7\linewidth]{images/Bab_2/bab_2_2}
		\caption[Struktur Serat]{\small{Struktur umum serat optik}}
	\end{figure}
	
	Sinar yang masuk pada boundary core-cladding dengan sudut yang lebih besar daripada sudut kritis akan mengalami peristiwa total internal reflection dan akan dipandu melalui core tanpa mengalami pembiasan.
	Berdasarkan moda perambatannaya, serat optik dibagi menjadi dua jenis yaitu serat singlemode yang memiliki diameter core lebih kecil dan serat multimode yang memiliki diameter core lebih besar. 
	Tipe perambatan sinar pada core serat optik dibagi dua yaitu step-index dan graded-index.
	
	\begin{figure}[h!]
		\centering
		\captionsetup{justification=centering}
		\includegraphics[width=0.7\linewidth]{images/Bab_2/bab_2_3}
		\caption[Muka Gelombang Serat Optik]{\small{Bentuk Geometri, Profil Indeks Bias dan Tipe Perambatan sinar pada MMF Step, SMF,dan MMF Graded}}
	\end{figure}


	\subsection{Multimode Interference (MMI)}
	
	Multimode Interference (MMI) merupakan fenomena yang terjadi akibat adanya pemantulan cahaya secara berulang didalam susunan core dan cladding pandu gelombang.
	Pemantulan yang berulang didalam core menyebabkan terjadinya interferensi internal, sehingga terjadi perubahan pola cahaya yang keluar dari core secara periodik. 
	Interferensi yang terjadi dapat secara konstruktif maupun destruktif bergantung pada profil indeks bias, jejari, radius, dan panjang gelombang operasi yang digunakan.
	Interferensi konstruktif yang terjadi secara periodik ini disebut sebagai self imaging.
	Fenomena self imaging didalam pandu gelombang multimode dapat dijelaskan menggunakan modal propagation analysis (MPA).
	
	\begin{figure}[h!]
		\centering
		\captionsetup{justification=centering}
		\includegraphics[width=0.7\linewidth]{images/Bab_2/bab_2_4}
		\caption[Pandu Gelombang SMS]{\small{Skema pandu gelombang multimode pada serat optik SMS}}
	\end{figure}

	Pada profil medan input (z = 0), moda yang berasal dari serat singlemode tereksitasi menjadi distribusi moda yang mungkin terpandu kedalam pandu gelombang serat multimode.
	Sedangkan pada profil medan (z=L), akan menghasilkan self imaging sebanyak n kali dengan jarak tertentu secara periodik (jarak reimaging) .
	Jarak self imaging ditentukan oleh konstanta propagasi antar moda yang berdekatan ($\beta_{m}$ dan $\beta_{m+1}$), dinyatakan sebagai berikut:
	
	\begin{align}
		L_{i} = 10 * \frac{\pi}{\beta_{m} + \beta_{m+1}}
	\end{align}
	
	\subsection{Efek Mekanis pada Serat Optik} 
	
	Efek mekanis disini adalah perlakuan mekanis terhadap serat yang dapat mempengaruhi daya yang dirambatkan oleh serat optik.
	Perlakuan yang dipilih disini adalah macro-bending dimana telah dilakukan penelitian bahwa macro-bending dapat mempengaruhi hasil trace OTDR \cite{Maharinda}.
	
	\subsection{Serat Optik sebagai Distributed Sensor}
	
	
	
	
%=============================================================================
% Daftar Pustaka
\newpage

	\section{Daftar Pustaka}
	
	\begin{center}
		\textbf{Daftar Pustaka}
	\end{center}
	
	\bibliographystyle{IEEEtran}
	\bibliography{/home/achmadi/Documents/BibTex/library.bib}

	
\end{document}
