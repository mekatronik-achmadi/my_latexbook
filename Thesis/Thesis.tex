\documentclass[12pt]{article}
\usepackage[utf8]{inputenc}
\usepackage[T1]{fontenc}
\usepackage{mathptmx}
\usepackage{geometry}
\usepackage{mathtools}
\usepackage[english]{babel}
\usepackage{graphicx}
\usepackage[figurename=Gambar]{caption}
\usepackage{hyperref}
\usepackage{minted}
\usepackage{setspace}
\usepackage{color}
\usepackage[explicit]{titlesec}
\usepackage{tocloft}
\usepackage{titletoc}
\usepackage{indentfirst}

%\color{white}
%\pagecolor{black}

\hypersetup{colorlinks,
	citecolor=black,
	filecolor=black,
	linkcolor=black,
	urlcolor=black
}

\geometry{
	a4paper,
	left=15mm,
	right=10mm,
	top=10mm,
	bottom=15mm,
}

\date{}

%=============================================================================
% Modified Items

\providecommand{\keywordid}[1]{\textit{Kata Kunci: } #1}
\providecommand{\keyworden}[1]{\textit{Keywords: } #1}

\titleformat{\section}{}{}{0pt}{}
\titleformat{\subsubsection}{\small \bfseries}{\thesubsection}{0pt}{}

\renewcommand{\cftsecleader}{\cftdotfill{\cftdotsep}}
\renewcommand{\cftsubsecleader}{\cftdotfill{\cftdotsep}}

\addto\captionsenglish{\renewcommand{\contentsname}{}}

\makeatletter
\let\latexl@section\l@section
\def\l@section#1#2{\begingroup\let\numberline\@gobble\latexl@section{#1}{#2}\endgroup}
\makeatother

\renewcommand{\cftsecfont}{\normalfont}
\renewcommand{\cftsecpagefont}{\normalfont}

%=============================================================================
% Document Part

\begin{document}
	
%=============================================================================
% Abstrak Indonesia
	
	\pagenumbering{roman}
	\setcounter{page}{2}
	
	\section{Abstrak}
	
	\begin{center}
		\onehalfspacing{\large{\textbf{STUDI EKSPERIMENTAL \textit{MULTI-POINT INTRUSION DETECTOR} BERBASIS OPTICAL TIME DOMAIN REFLECTOMETRY}}}
	\end{center}

	\vspace{10pt}
	
	\begin{flushleft}
		\hspace{150pt} Nama Mahasiswa : Achmadi \\
		\hspace{150pt} NRP \hspace{62pt}: 24 16 201 010 \\
		\hspace{150pt} Pembimbing \hspace{24pt}: Agus Muhamad Hatta, S.T, M.Si, Ph.D \\
	\end{flushleft}
	
	\vspace{10pt}

	\renewcommand{\abstractname}{ABSTRAK}
	\begin{abstract}
		Penyusup \textit{(intruder/attacker)} merupakan sebuah aktifitas yang tidak bisa dianggap untuk tidak mungkin terjadi apabila berkaitan dengan bangunan, area (perimeter), maupun objek lain dimana memiliki suatu nilai baik militer, ekonomi, maupun industri. 
		Tindak penyusupan memiliki similaritas dengan tindak pencurian dan menjadi awal dari tindak kriminal yang lebih jauh.
		Tindak penyusupan melalui proses melewati sistem pengawasan atau penjagaan.
		Untuk pencegahannya, maka sistem deteksi penyusup telah menjadi bagian penting dalam banyak sistem keamanan.
		Dibutuhkan sistem deteksi yang akurat dan respon yang cepat. Serat optik telah dikenal mampu menjadi transmisi data maupun sebagai sensor.
		Serat optik dapat digunakan sebagai distributed sensor yang mampu menggantikan banyak sensor tipe titik.
		OTDR telah dikenal sebagai salah satu metode karakteriasi serat optik.
		Melalui OTDR akan didapatkan events yang terjadi pada serat optik secara \textit{real-time}.
		Hasil \textit{trace} OTDR dapat digunakan untuk deteksi penyusup.
		Penelitian ini bertujuan untuk merancang sistem deteksi penyusup menggunakan serat optik dan OTDR yang mampu menemukan penyusup dalam multi-point.
		Dalam penelitian ini diusulkan sebuah studi eksperimental untuk mendapatkan rancang bangun sistem sensor terdistribusi berbasis serat optik dan OTDR untuk deteksi penyusup.
		Studi eksperimental disini divariasikan baik konfigurasi serat optik itu sendiri dan juga divariasikan bentuk distribusi sensor.
		Untuk mendapatkan hasil trace maka digunakan modul Mini-OTDR Anritsu MU909015C, sedangkan untuk eksperimen akan dibangun konstruksi pagar yang akan menjadi tempat instalasi sensor.
		Sebagai pengganti tindak intrusi maka diberikan gaya tekan kepada sensor dengan nilai dan jumlah yang telah ditentukan.
		Luaran yang diharapkan dari penelitian ini adalah hasil rancang bangun sensor instrusi terdistribusi dan rancang bangun algoritma untuk mendapatkan posisi multi-point dari tindak intrusi.   
	\end{abstract}

	\keywordid{Serat optik, OTDR, deteksi penyusup, trace, events, sensor terdistribusi}
	
%=============================================================================
\newpage
\thispagestyle{plain}
\mbox{}
	
%=============================================================================
% Abstrak English
	\newpage
	
	\section{Abstract}
	
	\begin{center}
		\onehalfspacing{\large{\textbf{EXPERIMENTAL STUDY OF MULTI-POINT INTRUSION DETECTOR BASED ON OPTICAL TIME DOMAIN REFLECTOMETRY}}}
	\end{center}
	
	\vspace{10pt}
	
	\begin{flushleft}
		\hspace{150pt} Nama Mahasiswa : Achmadi \\
		\hspace{150pt} NRP \hspace{62pt}: 24 16 201 010 \\
		\hspace{150pt} Pembimbing \hspace{24pt}: Agus Muhamad Hatta, S.T, M.Si, Ph.D \\
	\end{flushleft}
	
	\vspace{10pt}
	
	\renewcommand{\abstractname}{ABSTRACT}
	\begin{abstract}
		Intrusion by an intruder is an act that cannot be ignored due to a building, an area (perimeter), or any object regarding it’s either milltary, economical, or industrial values.
		Intrusion has high similarity to thieft crime and can be root of more crime act.
		Intrusion is an act that by-passing any survelleince or security system.
		For prevention, a detection system become essential to many security system.
		This detection system has to be highly accurate and fast respons.
		Optical fiber already known for it’s capabilities to both data transmission and as a distributed sensor.
		An distributed sensor mean single optical fiber section can replace many point-type sensors.
		OTDR already known as one of optical fiber characterization.
		Through OTDR, an events that occur on a optical fiber can be aqcuired by real-time.
		An trace result of OTDR can be used to intrusion detector.
		This research purposes is to get designs of an intrusion detector system using optical fiber and OTDR that can detect any multi-point intrusion.
		This research proposed an experimental study to get an intrusion detector using distributed sensor based optical fiber and OTDR.
		This experimental study proposed to test sensor variation in both optical fiber configuration and distributed sensor shape.
		To get trace result, this research use Anritsu MU909015C Mini OTDR module and for experiment, a fence construction is proposed as distributed sensor placement.
		As intrusion act, this research proposed to give a mechanical pressure to sensor with certain amount and magnitude.
		The expected output of this research are design of distributed sensor for intrusion detection and algorithm to get multi-point intrusion positions.
	\end{abstract}
	
	\keyworden{optical fiber, OTDR, instrusion detection, trace, events, distributed sensor}
	
%=============================================================================
\newpage
\thispagestyle{plain}
\mbox{}

%=============================================================================
% Halaman Pengesahan
\newpage

	\section{Lembar Pengesahan}
	
	\begin{center}
		\textbf{LEMBAR PENGESAHAN}
	\end{center}
	
	\begin{center}
		\textbf{DRAFT PROGRES I}
	\end{center}

	\vspace{10pt}

	\begin{flushleft}
		\hspace{50pt} Judul	: \onehalfspacing{\textbf{STUDI EKSPERIMENTAL MULTI-POINT INTRUSION DETECTOR BERBASIS OPTICAL TIME DOMAIN REFLECTOMETRY}}
	\end{flushleft}

	\begin{flushleft}
		\hspace{50pt} Oleh : Achmadi
	\end{flushleft}

	\begin{flushleft}
		\hspace{50pt} NRP : 24 16 201 010
	\end{flushleft}

	\vspace{20pt}
	
	\begin{center}
		\textbf{Telah diseminarkan pada :}
	\end{center}

	\vspace{10pt}

	\begin{flushleft}
		Hari \hspace{17pt}: \\
		Tanggal :\\
		Tempat \hspace{3pt}:  \\
	\end{flushleft}

	\vspace{20pt}
	
	\begin{center}
		\textbf{Mengetahui / menyetujui :}
	\end{center}

	
	\begin{center}
		Dosen Penguji \hspace{150pt} Dosen Pembimbing \\
		Prof. Dr. Ir. Sekartedjo, M.Sc \hspace{75pt} Agus M. Hatta, S.T, M.Si, Ph.D \\
		NIP. 19500402 1979 01 1 001 \hspace{85pt} NIP. 19780902 2003121 002 \\
	\end{center}

	\vspace{100pt}	
	
	\begin{flushleft}
		\hspace{50pt} Dr. rer. nat. Ir. Aulia M.T. Nasution., M.Sc \\
		\hspace{75pt} NIP. 19671117 199702 1 001
	\end{flushleft}

%=============================================================================
\newpage
\thispagestyle{plain}
\mbox{}

%=============================================================================
% Daftar Isi
\newpage

	\begin{center}
		\textbf{{\large Daftar Isi}}
	\end{center}
	
	\tableofcontents

%=============================================================================
\newpage
\thispagestyle{plain}
\mbox{}

%=============================================================================
% Daftar Gambar
\newpage

	\begin{center}
		\textbf{{\large Daftar Gambar}}
	\end{center}

%=============================================================================
\newpage
\thispagestyle{plain}
\mbox{}

%=============================================================================
% Daftar Tabel
\newpage

	\begin{center}
		\textbf{{\large Daftar Tabel}}
	\end{center}

%=============================================================================
\newpage
\thispagestyle{plain}
\mbox{}

%=============================================================================
% BAB I
\newpage

\pagenumbering{arabic}

	\setcounter{section}{0}
	\section{Pendahuluan}
	
	\begin{center}
		{\large \textbf{BAB I}} \\
		{\large \textbf{Pendahuluan}}
	\end{center}
	
	\subsection{Latar Belakang}
	
	Penyusupan (intrusion) merupakan sebuah aktifitas yang tidak bisa dianggap untuk tidak mungkin terjadi apabila berkaitan dengan bangunan, area (perimeter), maupun objek lain dimana memiliki suatu nilai baik militer, ekonomi, maupun industri\cite{Assets}.
	Pembahasan mengenai penyusup ini erat kaitan dalam pembahasan pencurian (theft) dalam bidang kriminologi, sehingga perilaku penyusupan memiliki kemiripan yang tinggi dengan fenomena pencurian \cite{Felson1998}.
	Kemiripan ini menempatkan definisi penyusupan tidak jauh terhadap definisi pencurian. Secara umum, penyusupan maupun pencurian adalah tindak kriminal yang melibatkan proses melewati maupun menerobos suatu sistem penjagaan atau pengawasan \cite{Chapman}.
	Menurut statistik international, tindak kriminal pencurian memang tidaklah setinggi kriminal lain yang berkaitan pembunuhan dan obat-terlarang \cite{Frate2010}.
	Namun demikian tetap dilakukan pencegahan karena penyusupan adalah awal dari beragam tindak kriminal lebih lanjut \cite{Nesbit}.
	
	Sistem pendeteksi penyusup saat ini telah mengalami perkembangan signifikan.
	Metode konvensional seperti patroli rutin/mendadak kini mulai terganti dengan sistem terintegrasi semisal Motion Detector, kamera pengintai, atau pagar listrik \cite{AFL2011}.

%=============================================================================
% Daftar Pustaka
\newpage

	\section{Daftar Pustaka}
	
	\begin{center}
		\textbf{Daftar Pustaka}
	\end{center}
	
	\bibliographystyle{IEEEtran}
	\bibliography{/home/achmadi/Documents/BibTex/library.bib}

	
\end{document}
