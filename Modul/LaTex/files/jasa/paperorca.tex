\documentclass[reprint]{JASA}

\begin{document}

\title[Bahasa Nyanyian Orca]{
	Kodifikasi Lagu \textit{Orcinus orca} sebagai Kajian Linguistik Non-Verbal
	dalam kerangka Biologi Evolusioner
}

\author{Achmadi}
\email{mekatronik.achmadi@gmail.com}
\author{Ammar Assyraf}
\affiliation{Teknik Fisika,  Institut Teknologi Surabaya, Surabaya, 60111, Indonesia}

\author{Seno Widya Manggala}
\email{manggala@kehidupan.edu}
\affiliation{Teknik Perkulian,  Universitas Kehidupan, Bogor, 11111, Indonesia}
\date{\today}

\begin{abstract}
	Mamalia dalam ordo \textit{Cetacea} dikenal sebagai mamalia \textit{pelagic} yang bersifat sosial.
	Dalam setiap komunitasnya, ditemukan perilaku nyanyian yang dapat dideteksi
	dalam air hingga puluhan kilometer.
	Dengan analisa spektrum audio, dapat ditemukan pola yang konsisten
	dan berulang untuk setiap grup keluarga.
	Pola tersebut dapat berbeda-beda pada grup keluarga berbeda dan kondisi emosi berbeda.
\end{abstract}

\maketitle

\section{\label{sec:1} Introduction}

Pertukaran vokal pada setiap individu yang telah dikenali polanya adalah
hasil belajar turun temurun di setiap grup keluarganya \cite{Ford1991}.
Dalam pengumpulan data bioakustik pada spesies ini, dapat dilakukan identifikasi
pola bahasa atau komunikasi menggunakan teknologi \textit{deep-learning} \cite{Bergler2019}.
Pola tersebut spesifik terhadap setiap grup keluarga yang bersifat matrinial \cite{Miller2004}.

\section{\label{sec:2} Section Two}

\bibliography{/home/achmaday/Documents/library.bib}

\end{document}
