\documentclass{book} % Definisi jenis dokumen

%%%%% Definisi paket-paket yang digunakan %%%%%
\usepackage[T1]{fontenc} % paket encoding huruf (wajib)
\usepackage[utf8]{inputenc} % paket encoding input (wajib)
\usepackage[yyyymmdd,hhmmss]{datetime} % paket tanggal-waktu
\usepackage{graphicx} % paket grafik/gambar
\usepackage[english]{babel} % paket modifikasi label/caption pada bahasa tertentu
\usepackage{geometry} % paket ukuran kertas dan margin
\usepackage{hyperref} % paket referensi url dan hyperref
\usepackage{minted} % paket lingkungan kode sumber
\usepackage{xcolor} % paket definisi warna

%%%%% Pengaturan ukuran kertas dan margin %%%%%
\geometry{
	a4paper,
	left=10mm,
	right=10mm,
	top=15mm,
	bottom=15mm,
}

%%%%% Pengaturan perintah informasi perangkat lunak %%%%%
\newcommand{\ShowOsVersion}{
	\immediate\write18{\unexpanded{foo=`uname -sro` && echo "${foo}" > tmp.tex}}
	\input{tmp}\immediate\write18{rm tmp.tex}
}

\newcommand{\ShowTexVersion}{
	\immediate\write18{\unexpanded{foo=`pdflatex -version | head -n1 | cut -d' ' -f1,2` && echo "${foo}" > tmp.tex}}
	\input{tmp}\immediate\write18{rm tmp.tex}
}

\newcommand{\ShowVimVersion}{
	\immediate\write18{\unexpanded{foo=`vim --version | head -n1 | cut -d'(' -f1` && echo "${foo}" > tmp.tex}}
	\input{tmp}\immediate\write18{rm tmp.tex}
}

%%%%% Mengganti label "Contents" ke "Daftar Isi" %%%%%
\addto\captionsenglish{\renewcommand{\contentsname}{Daftar Isi}}

%%%%% Pengaturan teks link url/hyperref %%%%%
\hypersetup{
	colorlinks=true, %set true if you want colored links
	linktoc=all,     %set to all if you want both sections and subsections linked
	linkcolor=blue,  %choose some color if you want links to stand out
	urlcolor=blue,   %url color
}

%%%%% Definisi warna baru %%%%%
\definecolor{LightGray}{gray}{0.95}

\begin{document}

    %%%%%%%%%%%%%%%%%%%%%%%%%%%%%%%%%%%%%%%%%%%%%%%%%%%%%%%%%%%%%%%%%

    \frontmatter % untuk halaman cover

    \begin{titlepage}
        \centering % untuk membuat tengah teks

        {
            \LARGE % pakai font besar
            \bf % pakai font BOLD
            Pengenalan LaTeX untuk Pemula
        }

        {\Large \bf Achmadi ST MT}
        \vfill % menambahkan ruang kosong vertikal

		\includegraphics[width=500pt]{images/leafcover}
        \vfill

        \raggedright
        \noindent Buku ini ditulis dengan:\\ % tanda \\ menambahkan garis baru
		OS : \ShowOsVersion \\
        Vim: \ShowVimVersion \\
		TeX : \ShowTexVersion \\
		Update: {\today} at \currenttime\\
    \end{titlepage}

    %%%%%%%%%%%%%%%%%%%%%%%%%%%%%%%%%%%%%%%%%%%%%%%%%%%%%%%%%%%%%%%%%

	\newpage % halaman baru

	\tableofcontents % daftar isi

	%%%%%%%%%%%%%%%%%%%%%%%%%%%%%%%%%%%%%%%%%%%%%%%%%%%%%%%%%%%%%%%%%

    \newpage
    \chapter{Disclaimer} % memulai chapter baru

    LaTeX adalah proyek typesetting yang dimulai oleh \href{https://en.wikipedia.org/wiki/Leslie_Lamport}{Leslie Lamport}
    sekitar 40 tahun lalu dan sekarang dilanjutkan oleh \href{https://www.latex-project.org/}{latex-project.org}.

    \bigskip % baris kosong pemisah paragraf

    Buku tutorial ini disarikan dari \href{https://latex-tutorial.com/tutorials}{latex-tutorial.com}
    dan \href{https://www.overleaf.com/learn/latex/Tutorials}{overleaf.com} dengan alih bahasa dan penambahan konten sesuai selera penulis.

    %%%%%%%%%%%%%%%%%%%%%%%%%%%%%%%%%%%%%%%%%%%%%%%%%%%%%%%%%%%%%%%%%

	\newpage
	\chapter{Penggunaan Buku}

    \section{Umum} % memulai section baru
	Buku ini dibuat dengan tujuan penggunaan utama sebagai panduan digital untuk mempermudah search dan copy-paste.
	Anda tidak perlu mencetak buku ini ke bentuk kertas.
	Seluruh navigasi buku ini diharapkan menggunakan klik ke hyperlink di Daftar Isi,
	atau menggunakan tampilan \textbf{Index} yang tersedia di \textbf{SideBar} program pembaca PDF yang anda gunakan.

	\section{Petunjuk}
	Beberapa petunjuk yang digunakan di buku ini:
    \begin{itemize} % memulai lingkungan daftar tanpa angka (bullet list)
		\item \textbf{Cetak Tebal}: Menginformasikan identifier (keyword, nama file, dst) yang berada di suatu paragraf
		\item \textit{Cetak Miring}: Bersama simbol panah (->) dan simbol lain, menginformasikan langkah-langkah klik menu/tombol.
		\item \textbf{TIPS:} Menginformasikan hal-hal yang dapat membantu atau pengetahuan tambahan.
		\item \textbf{PERINGATAN:} Menginformasikan hal-hal yang bener-benar harus diperhatikan.
	\end{itemize}

    \section{Penulisan Kode}
	Untuk penulisan kode, akan digunakan sebagai berikut:

    % lingkungan penulisan kode sumber
    \begin{minted}[frame=lines,framesep=2mm,fontsize=\normalsize,bgcolor=LightGray]{latex}
\begin{enumerate}
    \item Pertama
    \item Kedua
\end{enumerate }
    \end{minted}

    \section{Kode Sumber Dokumen}

    Seluruh isi buku ini juga ditulis dengan LaTeX, sehingga jika dibutuhkan contoh, dapat melihat
    kode sumber dokumen ini yang tersedia di \href{https://github.com/mekatronik-achmadi/my_latexbook/blob/master/Modul/LaTex/latex_beginner.tex}{GitHub}.

    %%%%%%%%%%%%%%%%%%%%%%%%%%%%%%%%%%%%%%%%%%%%%%%%%%%%%%%%%%%%%%%%%

    \newpage

    \mainmatter % pindah format halaman dari romawi ke angka (konten utama)
    \chapter{Introduksi}

    \section{Apa itu LaTeX}

    LaTeX bahasa markup (seperti HTML dan Markdown) untuk typeset (mengatur penulisan) dokumen.


\end{document}
