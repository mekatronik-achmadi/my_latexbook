\documentclass[12pt]{book}
\usepackage[utf8]{inputenc}
\usepackage[T1]{fontenc}
\usepackage{mathptmx}
\usepackage{geometry}
\usepackage{mathtools}
\usepackage[english]{babel}
\usepackage{graphicx}
\usepackage{subcaption}
\usepackage{stackengine}
\usepackage[os=win]{menukeys}
\usepackage{hyperref}
\usepackage{minted}
\usepackage{xcolor}
\usepackage{tikz}
\usepackage[yyyymmdd,hhmmss]{datetime}
\usepackage{etoolbox}
\usepackage[inline]{enumitem}
\usepackage{pdfpages}

\newcommand{\WindowsLogo}{\raisebox{-0.1em}{\includegraphics[height=0.8em]{images/logo/Windows_3_logo_simplified}}}
%\newcommand{\PowerLogo}{\raisebox{-0.1em}{\includegraphics[height=0.8em]{images/logo/power}}}
\newcommand{\WinKey}{\keys{\WindowsLogo}}
\newcommand{\PowerKey}{\keys{\PowerLogo}}

\patchcmd{\thebibliography}{\section*{\refname}}{}{}{}

\newcommand{\ShowOsVersion}{
	\immediate\write18{\unexpanded{foo=`uname -sro` && echo "${foo}" > tmp.tex}}
	\input{tmp}\immediate\write18{rm tmp.tex}
}

\newcommand{\ShowTexVersion}{
	\immediate\write18{\unexpanded{foo=`pdflatex -version | head -n1 | cut -d' ' -f1,2` && echo "${foo}" > tmp.tex}}
	\input{tmp}\immediate\write18{rm tmp.tex}
}

\addto\captionsenglish{\renewcommand{\contentsname}{Daftar Isi}}
\addto\captionsenglish{\renewcommand{\figurename}{Gambar}}

\hypersetup{
	colorlinks=true, %set true if you want colored links
	linktoc=all,     %set to all if you want both sections and subsections linked
	linkcolor=blue,  %choose some color if you want links to stand out
	urlcolor=blue,   %url color
}

\geometry{
	a4paper,
	left=10mm,
	right=10mm,
	top=10mm,
	bottom=15mm,
}

\title{\LARGE \bf
	Pengenalan MathWorks MATLAB dan Pemrogramannya\\
}

\author{}

\date{}

\hypersetup{citecolor=black}

\definecolor{LightGray}{gray}{0.95}

%\pagecolor[rgb]{0.1,0.1,0.1}
%\color[rgb]{1,1,1}

\begin{document}
	\frontmatter
	\maketitle
	
	%%%%%%%%%%%%%%%%%%%%%%%%%%%%%%%%%%%%%%%%%%%%%%%%%%%%%%%%%%%%%%%%%
	
	\newpage
	\tableofcontents
	
	%%%%%%%%%%%%%%%%%%%%%%%%%%%%%%%%%%%%%%%%%%%%%%%%%%%%%%%%%%%%%%%%%
	
	\newpage
	\chapter{Disclaimer}
	
	MATLAB adalah merek dagang dari The MathWorks, Inc.
	The MathWorks sendiri tidak menjamin akurasi isi buku ini.
	Penggunaan buku dalam kaitan perangkat lunak MATLAB tidak bermaksud promosi atau sponsor dari MathWorks.
	\\
	\\
	Konten buku ini disarikan dari buku \textit{"Chemical Engineering Computation with MATLAB"} oleh Yeong Koo Yeo,
	dipublikasikan oleh CRC Press tahun 2001.
	
	%%%%%%%%%%%%%%%%%%%%%%%%%%%%%%%%%%%%%%%%%%%%%%%%%%%%%%%%%%%%%%%%%
	
	\newpage
	\chapter{Penggunaan Buku}
	
	\section{Umum}
	Buku ini dibuat dengan tujuan penggunaan utama sebagai panduan digital.
	Anda tidak perlu mencetak buku ini ke bentuk kertas.
	Seluruh navigasi buku ini diharapkan menggunakan klik ke hyperlink di Daftar Isi,
	atau menggunakan tampilan \textbf{Index} yang tersedia di \textbf{SideBar} program pembaca PDF yang anda gunakan.
	
	\section{Petunjuk}
	Beberapa petunjuk yang digunakan di buku ini:
	\begin{itemize}
		\item \textbf{Cetak Tebal}: Menginformasikan identifier (keyword, variabel, fungsi, alamat, nama file, dst) yang berada di suatu paragraf
		\item \textbf{TIPS:} Menginformasikan hal-hal yang dapat membantu suatu aksi.
		\item \textbf{PERINGATAN:} Menginformasikan hal-hal yang layak diperhatikan.
		\item \textbf{CATATAN:} Menginformasikan hal-hal yang dapat menjadi pengetahuan tambahan.
	\end{itemize}

	\section{Penulisan Kode}
	Untuk penulisan kode, akan digunakan tiga bentuk:
	\begin{itemize}
		\item IN dan OUT. Berupa bagian kode, dengan:
		\begin{itemize}
			\item Baris dengan tanda \textbf{Prompt} (>>) adalah Input
			\item Baris dibawahnya tanpa ada tanda \textbf{Prompt} (>>) adalah contoh Output
		\end{itemize}
	
		\begin{minted}[frame=lines,framesep=2mm,fontsize=\small,bgcolor=LightGray]{matlab}
>> input
output
		\end{minted}
	
		\item IN saja. Hanya sebagai input dengan ditandai \textbf{Prompt} (>>).
		Contoh Output disini tidak ditampilkan.
		
		\begin{minted}[frame=lines,framesep=2mm,fontsize=\small,bgcolor=LightGray]{matlab}
>> input
		\end{minted}
	
		\item Script/Function. Kode ditulis sebagai file script atau function tanpa ada tanda \textbf{Prompt} (>>) sama sekali.
		
		\begin{minted}[frame=lines,framesep=2mm,fontsize=\small,bgcolor=LightGray]{matlab}
function y=tambah(a,b)
	y = a + b
end
		\end{minted}
\end{itemize}
	
	%%%%%%%%%%%%%%%%%%%%%%%%%%%%%%%%%%%%%%%%%%%%%%%%%%%%%%%%%%%%%%%%%
	
	\newpage
	\mainmatter
	\chapter{Program MATLAB}
	
	\section{Memulai Program}
	\subsection{Windows}
	\subsection{GNU/Linux}
	\subsection{MacOS}
	
	\section{Bagian Antar Muka}
	\subsection{Default Layout}
	\subsection{Toolbar}
	\subsection{Address Bar}
	\subsection{Current Folder}
	\subsection{Command Window}
	\subsubsection{Prompt}
	\subsection{Workspace}
	\subsection{Editor}
	
	
\end{document}